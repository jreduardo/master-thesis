% ------------------------------------------------------------------------
% ------------------------------------------------------------------------
% abnTeX2: Modelo de Projeto de pesquisa em conformidade com
% ABNT NBR 15287:2011 Informação e documentação - Projeto de pesquisa -
% Apresentação
% ------------------------------------------------------------------------
% ------------------------------------------------------------------------

%-------------------------------------------
% Consigurações da classe do documento
\documentclass[
    oldfontcommands,
    11pt,
    openright,
    twoside,
    a4paper,
    english,
    brazil
]{abntex2}\usepackage[]{graphicx}\usepackage[]{color}
%% maxwidth is the original width if it is less than linewidth
%% otherwise use linewidth (to make sure the graphics do not exceed the margin)
\makeatletter
\def\maxwidth{ %
  \ifdim\Gin@nat@width>\linewidth
    \linewidth
  \else
    \Gin@nat@width
  \fi
}
\makeatother

\definecolor{fgcolor}{rgb}{0.345, 0.345, 0.345}
\newcommand{\hlnum}[1]{\textcolor[rgb]{0.686,0.059,0.569}{#1}}%
\newcommand{\hlstr}[1]{\textcolor[rgb]{0.192,0.494,0.8}{#1}}%
\newcommand{\hlcom}[1]{\textcolor[rgb]{0.678,0.584,0.686}{\textit{#1}}}%
\newcommand{\hlopt}[1]{\textcolor[rgb]{0,0,0}{#1}}%
\newcommand{\hlstd}[1]{\textcolor[rgb]{0.345,0.345,0.345}{#1}}%
\newcommand{\hlkwa}[1]{\textcolor[rgb]{0.161,0.373,0.58}{\textbf{#1}}}%
\newcommand{\hlkwb}[1]{\textcolor[rgb]{0.69,0.353,0.396}{#1}}%
\newcommand{\hlkwc}[1]{\textcolor[rgb]{0.333,0.667,0.333}{#1}}%
\newcommand{\hlkwd}[1]{\textcolor[rgb]{0.737,0.353,0.396}{\textbf{#1}}}%
\let\hlipl\hlkwb

\usepackage{framed}
\makeatletter
\newenvironment{kframe}{%
 \def\at@end@of@kframe{}%
 \ifinner\ifhmode%
  \def\at@end@of@kframe{\end{minipage}}%
  \begin{minipage}{\columnwidth}%
 \fi\fi%
 \def\FrameCommand##1{\hskip\@totalleftmargin \hskip-\fboxsep
 \colorbox{shadecolor}{##1}\hskip-\fboxsep
     % There is no \\@totalrightmargin, so:
     \hskip-\linewidth \hskip-\@totalleftmargin \hskip\columnwidth}%
 \MakeFramed {\advance\hsize-\width
   \@totalleftmargin\z@ \linewidth\hsize
   \@setminipage}}%
 {\par\unskip\endMakeFramed%
 \at@end@of@kframe}
\makeatother

\definecolor{shadecolor}{rgb}{.97, .97, .97}
\definecolor{messagecolor}{rgb}{0, 0, 0}
\definecolor{warningcolor}{rgb}{1, 0, 1}
\definecolor{errorcolor}{rgb}{1, 0, 0}
\newenvironment{knitrout}{}{} % an empty environment to be redefined in TeX

\usepackage{amsmath}
\usepackage{alltt}
\usepackage[margin=2cm]{geometry}
\usepackage{bookman}

%-------------------------------------------
% Pacotes para uso geral
\usepackage{lmodern}
\usepackage[T1]{fontenc}
\usepackage[utf8]{inputenc}
\usepackage{indentfirst}
\usepackage{color}
\usepackage{graphicx}
\usepackage{microtype}
\usepackage{amsmath, amssymb, amstext}
\usepackage{bm}
\usepackage{setspace}
\usepackage{enumitem}
\usepackage[brazilian]{cleveref}

\usepackage{caption}
\captionsetup{
  justification=raggedright,
  singlelinecheck=false,
  font=small
}

%-------------------------------------------
% Pacotes para elaboração do cronograma
\usepackage{tikz}
\usepackage{pdflscape}
\usepackage{pgfgantt}
\usetikzlibrary{backgrounds}
\definecolor{done}{RGB}{120, 180, 120}
\definecolor{do}{RGB}{180, 120, 120}
\newcommand{\AfterMySpacing}{%
  \vskip\dp\strutbox
  \prevdepth\dp\strutbox}
\newenvironment{MySpacing}[1]
{\par\nointerlineskip
  \vskip\baselineskip
  \def\baselinestretch{#1}\@currsize
  \vskip-\arraystretch\ht\strutbox\relax
}
{\par\nointerlineskip
  \vskip-\arraystretch\dp\strutbox\relax
  \ignorespacesafterend\aftergroup\AfterMySpacing}

\makeatother

%-------------------------------------------
% Pacotes de citações
\usepackage{natbib}
% \bibliographystyle{humannat}
\bibliographystyle{agsm}

% \usepackage[brazilian,hyperpageref]{backref}
% \usepackage[alf]{abntex2cite}
% \renewcommand{\backrefpagesname}{Citado na(s) página(s):~}
% \renewcommand{\backref}{}
% % Define os textos da citação
% \renewcommand*{\backrefalt}[4]{
%   \ifcase #1 %
%   Nenhuma citação no texto.%
%   \or
%   Citado na página #2.%
%   \else
%   Citado #1 vezes nas páginas #2.%
%   \fi}%

%-------------------------------------------
% Informações de dados para CAPA e FOLHA DE ROSTO
\titulo{Modelos Flexíveis para Análise de Dados de Contagem}
\autor{Eduardo Elias Ribeiro Junior}
\local{Piracicaba \\ Estado de São Paulo -- Brasil}
\data{Junho / 2018}
\instituicao{
  Universidade de São Paulo \par
  Escola Superior de Agricultura ``Luiz de Queiroz'' \par
  Departamento de Ciências Exatas
}
\orientador[Orientadora:]{
  Profa. Dra. Clarice Garcia Borges Demétrio}
\tipotrabalho{Projeto de Pesquisa}
\preambulo{Plano de pesquisa apresentado ao serviço de pós-graduação da
  ESALQ/USP em conformidade com as normas do programa de pós-graduação
  em Estatística e Experimentação Agronômica.}

%-------------------------------------------
% Configurações de aparência do PDF final
% Alterando o aspecto da cor azul
\definecolor{blue}{RGB}{41,5,195}

% informações do PDF
\makeatletter
\hypersetup{
  pdftitle={\@title},
  pdfauthor={\@author},
  colorlinks=true,
  linkcolor=blue,
  citecolor=blue,
  filecolor=magenta,
  urlcolor=blue,
  bookmarksdepth=4
}
\addto\captionsbrazil{
  \renewcommand{\bibname}{REFER\^ENCIAS}
}
\makeatother

%-------------------------------------------
% Espaçamentos entre linhas e entre parágrafos parágrafos
\setlength{\parindent}{1.3cm}
\setlength{\parskip}{0.2cm}
\renewcommand{\baselinestretch}{1}

% Compila o índice
\makeindex

%-------------------------------------------
% Início do documento
\IfFileExists{upquote.sty}{\usepackage{upquote}}{}
\begin{document}
\selectlanguage{brazil}

% ----------------------------------------------------------
% ELEMENTOS PRÉ-TEXTUAIS
% ----------------------------------------------------------

\imprimircapa
\imprimirfolhaderosto

% ---

% ---
% inserir o sumario
% ---
\pdfbookmark[0]{\contentsname}{toc}
\tableofcontents*
\cleardoublepage
% ---





% ----------------------------------------------------------
% ELEMENTOS TEXTUAIS
% ----------------------------------------------------------
\textual

% ----------------------------------------------------------
% Introdução
% ----------------------------------------------------------
\chapter{Introdução}
\label{cha:introducao}

A classe dos modelos lineares generalizados (GLM) foi introduzida por
\citet{Nelder1972} contemplando modelos para análise de dados normais e
não normais em uma mesma teoria. \citet{Wedderburn1974} generalizou essa
classe para os modelos de quase-verossimilhança, em que há apenas a
suposição de primeiro e segundo momentos, ou seja, descreve-se apenas a
relação entre a média e a variância da variável resposta. Embora a
classe estendida para quase-verossimilhança seja mais flexível do que os
GLM's, na maioria das situações não é possível recuperar a distribuição
da variável resposta \citep{Paula2013}.

Os modelos lineares generalizados podem ser ajustados por um algoritmo
Newton-escore eficiente e há muitos softwares estatísticos que
implementam facilidades para isso. Além disso, há excelentes
contribuições na literatura da área como \citet{McCullagh1989,
  Venables2002} e \citet{Dobson2008}. Assim, os modelos lineares
generalizados se tornaram métodos proeminentes na análise de dados em
estatística aplicada.

Apesar da flexibilidade dos modelos lineares generalizados, a relação
média--variância, determinada pela função de variância $V(\mu_i)$, pode
ser bastante restritiva, principalmente nos casos de dados binomiais, de
contagens e de tempo até o evento \citep{Molenberghs2010,
  Molenberghs2017}. Para contagens, por exemplo, o modelo Poisson tem
função de variância $V(\mu_i) = \mu_i$, isto é, a esperança e a
variância da distribuição são iguais para a $i$-ésima observação,
característica conhecida como equidispersão. Porém, na prática, os dados
podem apresentar características de superdispersão (média $<$ variância)
e subdispersão (média $>$ variância), que tornam o modelo Poisson
inadequado. Nesses casos, os coeficientes ainda podem ser estimados
consistentemente, porém os erros padrões das estimativas são incorretos
\citep{Winkelmann1994, Hinde1998}. Em particular, um modelo Poisson
ajustado a contagens superdispersas leva à subestimação dos erros
padrões das estimativas; e para contagens subdispersas os erros padrões
são superestimados.

O caso mais comum de falha da suposição de equidispersão e,
consequentemente com um maior número de abordagens possíveis, é a
superdispersão. \citet{Hinde1998} discutem diversas razões que podem
levar à variabilidade extra Poisson. Dentre elas, citam-se amostragem
por aglomerados, heterogeneidade das unidades amostras, dados em nível
agregado e correlação entre observações. Os processos que reduzem a
variabilidade das contagens, abaixo do estabelecido pela Poisson, não
são tão conhecidos quanto os que produzem variabilidade extra. Pela
mesma razão, são poucas as abordagens descritas na literatura capazes de
tratar subdispersão. Uma das causas da subdispersão pode ser a violação
do processo Poisson, em que o tempo entre eventos pode não ser mais
exponencialmente distribuído; esse processo motiva a classe dos modelos
de dependência de duração \citep{Winkelmann1995}. Outra possível causa
da subdispersão está relacionada à obtenção de estatísticas de ordem da
variável resposta, por exemplo tomar o máximo das contagens observadas
\citep{Steutel1989}.

Na \Cref{fig:processo-pontual}, são apresentados diferentes tipos de
processos pontuais para ilustrar o processo Poisson em um espaço
bidimensional. Cada ponto representa a ocorrência de um evento e cada
quadrado, delimitado pelas linhas pontilhadas, representa a unidade (ou
domínio) na qual se conta o número de eventos (como variável
aleatória). A \Cref{fig:processo-pontual}(a), representa a situação
de dados de contagem equidispersos. As ocorrências dos eventos se
dispõem aleatoriamente. Na \Cref{fig:processo-pontual}(b), o padrão
já se altera, tem-se a representação do caso de superdispersão. Nesse
cenário, formam-se aglomerados que deixam parcelas com contagens muito
elevadas e parcelas com contagens baixas. Uma possível causa desse
padrão se dá pelo processo de contágio (e.g. contagem de casos de uma
doença contagiosa, contagem de frutos apodrecidos). Na
\Cref{fig:processo-pontual}(c), ilustra-se o caso de subdispersão, em
que as ocorrências se dispõem uniformemente no espaço. Agora, as
contagens de ocorrências nas parcelas variam bem pouco. Ao contrário da
superdispersão, uma causa interpretável seria o oposto de contágio, a
repulsa, ou seja, uma ocorrência causa a repulsa de outras ocorrências
em seu redor (e.g. contagem de árvores) ou ainda a competição
(e.g. contagem de animais territoriais ou que disputam por território).

\begin{knitrout}
\definecolor{shadecolor}{rgb}{0.969, 0.969, 0.969}\color{fgcolor}\begin{figure}[!htb]

{\centering \includegraphics[width=\textwidth]{figure/processo-pontual-1} 

}

\caption[Ilustração de diferentes tipos de processos pontuais]{Ilustração de diferentes tipos de processos pontuais. Da direita para esquerda têm-se processos sob padrões aleatório, aglomerado e uniforme.}\label{fig:processo-pontual}
\end{figure}


\end{knitrout}

As alternativas paramétricas para análise de dados na forma de contagens
não equidipersas estão, geralmente, relacionadas às causas da não
equidispersão. Para superdispersão, destacam-se os modelos mistos, que
incluem efeitos aleatórios em nível de observação, considerando a
heterogeneidade não observada. Um exemplo bem conhecido dessa prática é
o modelo Poisson com efeitos aleatórios gama, que resulta no modelo
binomial negativo. Porém, outras escolhas para a distribuição dos
efeitos aleatórios podem ser tomadas como, por exemplo, o modelo
Poisson-Tweedie \citep{Bonat2018} e seus casos particulares
Poisson-inversa Gaussiana (PIG) e Neyman-Type A assumem que os efeitos
aleatórios são Tweedie, Gaussiano inverso e Poisson distribuídos,
respectivamente.

Além dos modelos mistos, alguns modelos alternativos que modelam sub e
superdispersão têm merecido atenção da comunidade estatística. Dentre
eles destacam-se: o modelo \textit{Gamma-Count} \citep{Winkelmann1995,
  Zeviani2014}, que considera a distribuição gama para o tempo entre
eventos; o modelo COM-Poisson \citep{Shmueli2005, Sellers2010}, em que a
razão de recorrência das probabilidades é não linear por meio da adição
de um parâmetro de dispersão; e o modelo Poisson generalizado
\citep{Consul1992, Zamani2012}, uma generalização biparamétrica do
modelo Poisson.

Embora as distribuições citadas sejam flexíveis para lidar com
diferentes níveis de dispersão, há situações em que o delineamento do
experimento sugere uma estrutura de covariância entre observações
induzida por um processo hierárquico de casualização ou amostragem e
portanto, devem ser consideradas no modelo. São casos assim os
experimentos em parcelas subdivididas e experimentos com medidas
repetidas ou longitudinais.  Tais estruturas estabelecem modelos com
efeitos não observáveis e isso pode ser incorporado no modelo de
regressão com a inclusão de efeitos aleatórios em nível de grupos
experimentais; tais modelos são denominados modelos lineares
generalizados mistos (GLMM) \citep{Molenberghs2005}.

Para dados de contagem hierárquicos, o modelo GLMM Poisson assume que as
contagens são distribuídas condicionalmente como Poisson e que os
efeitos aleatórios são normalmente distribuídos. Embora esse modelo se
adeque a diversas situações, casos em que as contagens condicionais não
são equidispersas são pouco explorados na
literatura. \citet{Molenberghs2007, Molenberghs2010, Molenberghs2017}
apresentam uma nova família de modelos que consideram efeitos aleatórios
normais, para contemplar a estrutura hierárquica e conjugados, para
as contagens condicionalmente superdispersas. No entanto, modelos
hierárquicos mais flexíveis, capazes de modelar diferentes níveis de
dispersão das contagens condicionais aos efeitos aleatórios, não são
consolidadas na literatura e são escassas suas aplicações.

% ----------------------------------------------------------
% Objetivos
% ----------------------------------------------------------
\chapter{Objetivos}
\label{cha:objetivos}

\section{Objetivo Geral}
\label{sec:objetivosgerais}

Contribuir para a área de análise de dados na forma de contagens,
explorando distribuições probabilísticas que podem ser adotadas na
estrutura aleatória de modelos de regressão estendidos para análise de
dados hierárquicos e heterogêneos.

\section{Objetivos Específicos}
\label{sec:objetivosespecificos}

Como objetivos específicos, têm-se:

\begin{enumerate}[label=(\roman*)]
\item Propor uma nova reparametrização para o modelo COM-Poisson,
  avaliar a flexibilidade do modelo, as propriedades dos estimadores e
  apresentar aplicações;
\item Revisar distribuições capazes de modelar diferentes níveis de
  dispersão, destacando suas diferenças usando-se estudos de caso e
  dados simulados;
\item Propor e avaliar as propriedades de um modelo linear generalizado
  duplo, em que o parâmetro de dispersão das distribuições para dados na
  forma de contagens também é modelado com covariáveis; e
\item Propor e avaliar as propriedades de um modelo linear generalizado
  misto, em que a distribuição para a variável aleatória de contagem,
  condicional aos efeitos aleatórios, é flexível para lidar com
  diferentes níveis de dispersão.
\end{enumerate}

Todos os conjuntos de dados e as implementações computacionais
utilizadas, serão disponibilizadas para que toda a pesquisa seja
facilmente reproduzível. Como forma de divulgação científica, almeja-se
a produção e publicação de artigos científicos em periódicos na área de
análise de dados e modelagem estatística.

% ----------------------------------------------------------
% Materiais e Métodos
% ----------------------------------------------------------
\chapter{Estudos de caso}
\label{cha:estudosdecaso}

Nesse capítulo, são apresentados cinco estudos de caso, em que a
variável resposta se apresenta como contagem e, portanto, podem ilustrar
aplicações dos modelos a serem propostos.

\section{Desfolha artificial em capulhos de algodão}
\label{sec:cotton}

No cultivo de plantas de algodão, algumas pragas, doenças,
fitotoxicidade por substâncias químicas, granizo e certas injúrias
mecânicas são os principais agentes de desfolha, causando prejuízos na
produtividade. A fim de explorar os efeitos da redução da área foliar na
cultura do algodão \textit{Gossypium hirsutum}, \citet{Silva2012}
conduziram um experimento fatorial $5$ (níveis de desfolha) $\times $ 5
(estágios fenológicos) no delineamento inteiramente casualizados com
cinco repetições e condições controladas em uma casa de vegetação. A
variável de interesse foi o número de capulhos de algodão em cada
unidade experimental (vaso com duas plantas). Na
\Cref{fig:desc-cotton}, são apresentados os números de capulhos
observados em cada combinação entre estágio fenológico e nível de
desfolha.

\begin{knitrout}
\definecolor{shadecolor}{rgb}{0.969, 0.969, 0.969}\color{fgcolor}\begin{figure}[!htb]

{\centering \includegraphics[width=\textwidth]{figure/desc-cotton-1} 

}

\caption[Dispersão do número de capulhos observados para cada nível de desfolha artificial e estágio fenológico da planta]{Dispersão do número de capulhos observados para cada nível de desfolha artificial e estágio fenológico da planta. As linhas contínuas são curvas suaves estimadas pelo algoritmo \textit{lowess}.}\label{fig:desc-cotton}
\end{figure}


\end{knitrout}

Diversas alternativas já foram propostas para análise desse conjunto de
dados. \citet{Zeviani2014} utilizaram o modelo \textit{Gamma-Count},
\citet{Huang2017} e \citet{RibeiroJr2018} utilizaram reparametrizações
do modelo COM-Poisson e \citet{Bonat2018} apresentaram os resultados da
aplicação do modelo Poisson-Tweedie estendido. A notável relevância
desse conjunto de dados é a característica de subdispersão em uma
estrutura fatorial.

\section{Dose potássica e umidade do solo na produtividade de soja}
\label{sec:soybean}

Os solos tropicais, normalmente pobres em potássio (K), quando
cultivados com soja (\textit{Glycine max L.}) demandam adubação
potássica para obtenção de produtividades satisfatórias. Em um
experimento fatorial $(5 \times 3)$ conduzido em casa de vegetação no
delineamento de blocos casualizados, \citet{Serafim2012} avaliaram a
influência de doses de potássio ($0, 30, 60, 120$ e $180$ mg dm$^{-3}$)
combinadas com diferentes níveis de umidade do solo ($37,5; 50$ e $62,5$
\% do volume total de poros) no número de grãos e de vagens viáveis. Na
\Cref{fig:desc-soybean}, são apresentados os valores contabilizados de
cada variável de interesse, grãos de soja e vagens viáveis, para cada um
dos tratamentos, combinação entre umidade do solo e adubação com
potássio.



\citet{RibeiroJr2018} analisaram o número de grãos por parcela com sua
proposta de reparametrização do modelo COM-Poisson. A característica
interessante desse conjunto é que há diferentes níveis de dispersão em
cada contagem mensurada. Para o número de grãos por parcela, por
exemplo, \citet{RibeiroJr2018} mostram que há uma variabilidade extra
Poisson. Outra característica ainda não contemplada na análise desses
dados é o comportamento não linear das contagens, em particular, modelos
lineares com platô apresentam-se como bastante adequados, uma vez que se
espera haver um particular nível, a partir do qual, não se tem mais
influência de adubação potássica nas contagens.

\section{Ensaio clínico em pacientes epilépticos}
\label{sec:epilepsy}



Epilepsia é uma doença que se manifesta por crises de perda da
consciência, acompanhadas de convulsões, que ocorrem em intervalos
irregulares de tempo. A manipulação de drogas para diminuir as
convulsões epilépticas é um campo de interesse em
medicina. \citet{Faught1996} apresentam um estudo aleatorizado,
multicêntrico e duplo-cego para avaliação de uma nova droga
anti-epiléptica (AED), topiramate, em comparação com um falso placebo,
combinados de uma ou duas outras AEDs. A aleatorização dos pacientes
ocorreu após um período de 12 semanas de estabilização. Devidamente
aleatorizados, 45 pacientes foram medicados periodicamente com o placebo
e 44 com a AED topiramate. Os pacientes foram acompanhados semanalmente,
registrando-se o número de convulsões da respectiva semana. Na
\Cref{fig:desc-epilepsy}, são apresentados os perfis do número de
convulsões epilépticas para cada paciente, bem como os perfis médio e
mediano. Como destacam \citet{Molenberghs2007}, é interessante notar a
forte assimetria à direita na distribuição do número de convulsões, a
presença de valores extremos e a redução no número de pacientes
acompanhados no decorrer do estudo. Além disso, observa-se que há
excesso de contagens nulas (semanas em que não foram observadas
convulsões epilépticas em um determinado paciente) que representam
\text{33,122}\% dos
dados observados.

\begin{knitrout}
\definecolor{shadecolor}{rgb}{0.969, 0.969, 0.969}\color{fgcolor}\begin{figure}[!htb]

{\centering \includegraphics[width=\textwidth]{figure/desc-epilepsy-1} 

}

\caption[Perfis do número de convulções de cada paciente dos grupos tratamento e placebo]{Perfis do número de convulções de cada paciente dos grupos tratamento e placebo. As linhas em preto representam os perfis médio (linha contínua) e mediano (linha pontilhada).}\label{fig:desc-epilepsy}
\end{figure}


\end{knitrout}

Esse conjunto de dados é interessante, pois traz observações
multivariadas para cada indivíduo, isto é, para cada $i$-ésimo indivíduo
registram-se $n_i$ observações ($n_i$, número de semanas em que o
$i$-ésimo indivíduo foi acompanhado). Isto traz desafios para análise,
que são frequentemente superados pela inclusão de efeitos aleatórios a
fim de induzir correlação entre observações de um mesmo
indivíduo. \citet{Molenberghs2005} utilizam esse conjunto de dados como
exemplo para a aplicação dos modelos lineares generalizados Poisson pela
abordagem condicional (efeitos aleatórios) e marginal (equações de
estimação generalizadas). \citet{Molenberghs2007} e
\citet{Molenberghs2010} também utilizam esses dados como exemplo de
aplicação de um modelo Poisson misto, porém os autores propõem a
inclusão de duas fontes de efeitos aleatórios, para ajustar a
superdispersão inerente à variável de contagem, e para incorporar a
estrutura longitudinal do estudo. Para análise desse conjunto de dados,
pretende-se adotar uma distribuição mais flexível para o número de
convulsões de cada indivíduo ao invés de incluir efeitos aleatórios em
nível de observação.

\section{Substratos alternativos para crescimento de bromélias}
\label{sec:bromelia}

Xaxim é um substrato utilizado no cultivo de bromélias e orquídeas, cujo
comercialização foi proibida em 2001. Desde então, há pesquisas em
botânica para propor substratos alternativos ao Xaxim no cultivo de
bromélias, orquídeas e outras epífitas. Esse conjunto de dados provém de
um experimento aleatorizado em 4 blocos, cujo objetivo foi avaliar 5
diferentes recipientes de substratos alternativos para bromélias. Todos
os tratamentos continham turfa e perlita e se diferenciavam no terceiro
componente: casca de Pinus, casca de Eucaliptos, Coxim, fibra de coco e
Xaxim. A variável de interesse foi o número de folhas por unidade
experimental (pote com inicialmente 8 plantas), que foi registrado
diariamente durante 6 dias após a plantação.

\begin{knitrout}
\definecolor{shadecolor}{rgb}{0.969, 0.969, 0.969}\color{fgcolor}\begin{figure}[!htb]

{\centering \includegraphics[width=\textwidth]{figure/desc-bromelia-1} 

}

\caption[(a) Perfis do número de folhas observadas para cada bloco em cada componente dos substratos alternativos, as linhas em preto representam os perfis médios]{(a) Perfis do número de folhas observadas para cada bloco em cada componente dos substratos alternativos, as linhas em preto representam os perfis médios. (b) Dispersão das médias e variâncias amostrais, na escala logarítmica, calculadas sobre os blocos em cada tempo e tratamento, a linha pontilhada representa a reta média $=$ variância (equidispersão) e a contínua representa um ajuste de mínimos quadrados.}\label{fig:desc-bromelia}
\end{figure}


\end{knitrout}

Na \Cref{fig:desc-bromelia}(a), são apresentados os perfis
individuais do número de folhas para cada bloco em cada tratamento e
seus respectivos perfis médios. Nota-se um comportamento não linear
(sigmoidal) dos perfis em cada tratamento e uma pouca variabilidade
entre blocos em um mesmo tempo e tratamento. Essa última característica
é destacada na \Cref{fig:desc-bromelia}(b), onde os logaritmos das
médias e variâncias são apresentados em um gráfico de dispersão. Para
todas as combinações de tempo e tratamento, as variâncias do número de
folhas obtidas dos quatro blocos são bastante menores do que as
respectivas médias. Com isso, esse conjunto de dados apresenta alguns
desafios para análise, pois não se encontraram na literatura, exemplos
de análise de dados com as características de subdispersão em estudos
longitudinais.

\section{Infestação de mosca-branca em cultura de soja}

A mosca-branca \textit{Bemisia tabaci} é praga de diversas culturas
sendo capaz de se alimentar de mais de 500 espécies de vegetais. Na
cultura de soja, essa espécie de mosca-branca era pouco citada até a
primeira década de 2000. Porém, períodos de seca e o uso intensivo de
inseticidas e de fungicidas não selecionados afetaram seus inimigos
naturais e, consequentemente, contribuíram para o aumento da importância
econômica dessa praga em agronomia.

\begin{knitrout}
\definecolor{shadecolor}{rgb}{0.969, 0.969, 0.969}\color{fgcolor}\begin{figure}[!htb]

{\centering \includegraphics[width=\textwidth]{figure/desc-whitefly-1} 

}

\caption[(a) Perfis das contagens de ninfas observadas para cada bloco em cada uma das quatro cultivares BRS, as linhas em preto representam os perfis médios]{(a) Perfis das contagens de ninfas observadas para cada bloco em cada uma das quatro cultivares BRS, as linhas em preto representam os perfis médios. (b) Dispersão das médias (eixo $x$) e variâncias (eixo $y$) amostrais, na escala logarítmica, calculadas sobre os blocos em cada tempo e cultivar para cada variável de interesse, a linha pontilhada representa a reta média $=$ variância (equidispersão) e a contínua representa um ajuste de mínimos quadrados.}\label{fig:desc-whitefly}
\end{figure}


\end{knitrout}

Sob essa motivação, \citet{Suekane2013} conduziram um experimento sob o
delineamento de blocos para avaliar o número de ninfas de mosca-branca
em dez diferentes cultivares de soja. As plantas das diferentes
cultivares de soja foram cultivadas em vasos contendo duas plantas. No
início do experimento, houve infestação artificial de adultos de
mosca-branca na casa de vegetação. Unidades amostrais, contendo dois
vasos, foram avaliadas antes da infestação e nos dias 8, 13, 22, 31 e
38, quando se mensurou o número de ninfas da mosca-branca em um folíolo
do terço superior, médio e inferior da planta. Nessa pesquisa,
inicialmente, pretende-se utilizar somente as cultivares BRS das 10
cultivares experimentadas, devido às contagens nessas cultivares se
comportarem de forma similar e ao interesse na comparação das cultivares
transgênicas BRS produzidas pela Embrapa. Na
\Cref{fig:desc-whitefly}(a), são apresentados os perfis, das três
contagens realizadas, em cada bloco ao longo dos dias de infestação para
cada cultivar, e os respectivos perfis médios. O comportamento dos
perfis se mostra não linear, com um pico no número de ninfas ocorrendo
entre 10 e 20 dias de infestação. Além disso, ressalta-se que a
variabilidade das contagens é muito superior nos primeiros dias de
infestação. Na \Cref{fig:desc-whitefly}(b), apresentam-se as médias e
variâncias amostrais, em escala logarítmica, para cada variável de
interesse. Há uma forte evidência de superdispersão. Para análise desses
dados, modelos flexíveis para dados de contagens como Poisson-Tweedie,
COM-Poisson, \textit{Gamma-Count} e Poisson-generalizado, com a inclusão
de efeitos para modelar a estrutura longitudinal, são alternativas
bastante relevantes que ainda não são bem consolidadas na literatura. Em
adição, ressalta-se a possibilidade de modelar o parâmetro de dispersão
dessas distribuições com covariáveis, uma vez que se observou que a
variabilidade tem comportamento tempo-dependente.

% ----------------------------------------------------------
% Revisão de literatura
% ----------------------------------------------------------
\chapter{Metodologia}
\label{cha:metodologia}

Nesse capítulo são apresentados alguns modelos probabilísticos flexíveis
para dados discretos com suporte no conjunto dos números naturais, como
especificar os modelos de regressão e direções para estimação e
inferência.

\section{Distribuição Poisson}
\label{sec:poisson}

A distribuição Poisson é a principal referência para análise de dados na
forma de contagens. É uma distribuição pertencente à família exponencial
e tem interpretação como um modelo exponencial de dispersão
\citep{Jorgensen1997}. A distribuição de Poisson Po$(\mu)$, com média
$\mu$, tem função massa de probabilidade
\begin{equation}
  \label{eqn:pmf-poisson}
  \Pr(Y = y \mid \mu) = \frac{\mu^y \exp(-\mu)}{y!}, \quad y \in
  \mathbb{N},
\end{equation}
em que $\mu>0$. Pela função geradora de cumulantes da família
exponencial é fácil mostrar que E$(Y)=\text{Var(Y)}=\mu$. Como discutido
no \Cref{cha:introducao}, essa é uma particularidade bastante
restritiva da distribuição Poisson.  A linearidade das razões de
probabilidades consecutivas $\Pr(Y=y-1)/\Pr(Y=y) = y / \mu$ e a
relação da distribuição Poisson com a distribuição exponencial são
características que motivam classes de generalizações do modelo Poisson
e serão discutidas nas próximas seções.

Para especificar um modelo de regressão baseado na distribuição Poisson,
considere $y_i$ observações de uma variável Poisson e
$\bm{x}_i^\top = (x_{i1}, x_{i2}, \ldots, x_{ip})$ um vetor de
covariáveis conhecidas, $i=1,2,\ldots,n$.  O modelo de regressão Poisson
é especificado como
$$
Y_i \sim \text{Po}(\mu_i), \quad \text{em que} \quad \mu_i =
g^{-1}(\bm{x}_i^\top\bm{\beta}),
$$
em que $\bm{\beta}$ são parâmetros desconhecidos a serem estimados e
$g^{-1}(\cdot)$ uma função de ligação conhecida, por exemplo, logaritmo
(função de ligação canônica) e raiz quadrada.

\section{Distribuição COM-Poisson}
\label{sec:com-poisson}

A distribuição COM-Poisson é a principal representante da família de
distribuições Poisson ponderadas (WPD) \citep{DelCastillo1998}. Uma
variável aleatória $Y$ pertence à família WPD se sua função massa de
probabilidade puder ser escrita como
\begin{equation}
  \label{eqn:pmf-wpd}
  \Pr(Y = y) = \frac{w(y) \exp(-\lambda)\lambda^y}{
    y! \text{E}_\lambda[w(Y)]}, \quad y \in \mathbb{N},
\end{equation}
em que $\text{E}_\lambda(\cdot)$ é o valor médio calculado a partir de
uma variável aleatória Poisson de parâmetro $\lambda$, chamada de
constante de normalização; e $w(y)$ é uma função peso, não negativa e
tal que $\text{E}_\lambda[w(Y)]$ seja finita. A função peso
$w(y) \equiv w(y, \nu)$, pode depender de um parâmetro adicional de tal
forma que sub e superdispersão sejam abrangidas. Obtém-se a distribuição
COM-Poisson CMP$(\lambda, \nu)$ tomando-se $w(y, \nu) = (y!)^{1-\nu}$,
para $\nu \geq 0$ \citep{Sellers2012}. Sua função de probabilidade
assume a forma
\begin{equation}
  \label{eqn:pmf-compoisson}
  \Pr(Y = y) = \frac{\lambda^y \exp(-\lambda)}{
    (y!)^\nu \text{E}_\lambda[(Y!)^{1-\nu}]} =
  \frac{\lambda^y}{(y!)^\nu \text{Z}(\lambda, \nu)}, \quad
  \text{Z}(\lambda,\nu) = \sum_{j=0}^\infty \frac{\lambda^j}{(j!)^\nu},
\end{equation}
em que $\nu$ é dito como o parâmetro de dispersão tal que para $0<\nu<1$
e $\nu<1$ têm-se os casos de superdispersão e subdispersão,
respectivamente.  Para as distribuições pertencentes à família WPD e,
particularmente, para a distribuição COM-Poisson, a razão entre as
probabilidades de dois eventos consecutivos é dada por
\begin{equation}
  \frac{\Pr(Y = y - 1)}{\Pr(Y = y)} =
    \frac{y}{\lambda}\frac{w(y-1)}{w(y)} \overset{\text{CMP}}{\equiv}
    \frac{y^\lambda}{\lambda},
\end{equation}
enquanto que para a distribuição Poisson essa razão é $y /\lambda$,
correspondente a $w(y)$ constante ou, no caso COM-Poisson, $\nu=1$. Além
da Poisson ($\nu=1$), a distribuição COM-Poisson também tem como caso
particular a distribuição geométrica ($\nu=0$ e $\lambda<1$) e como caso
limite a distribuição Bernoulli ($\nu\to \infty$, sendo a probabilidade
de sucesso $\lambda/(\lambda+1)$).

Um inconveniente desse modelo é que os momentos média e variância, em
geral, não são obtidos em forma fechada. A partir de uma aproximação
para Z$(\lambda, \nu)$, \citet{Shmueli2005} e
\citet{Sellers2010} mostram que a esperança e a variância de uma
variável $Y \sim \text{CMP}(\lambda, \nu)$ podem ser aproximadas por
\begin{equation}
    \label{eqn:mean-aprox}
  \text{E}(Y) \approx \lambda^{1/\nu} - \frac{\nu - 1}{2\nu} \qquad
  \textrm{e} \qquad
  \text{Var}(Y) \approx \frac{\lambda^{1/\nu}}{\nu},
\end{equation}
que são particularmente acuradas para $\nu \leq 1$ ou $\lambda > 10^\nu$
\citep{Shmueli2005}. \citet{RibeiroJr2018} mostraram que a aproximação
para a média é acurada, diferentemente da aproximação para variância,
que perde acurácia quando $\nu<1$. Além disso, os autores indicam que a
acurácia das aproximações parece não ter relação com as regiões
$\nu \leq 1$ ou $\lambda > 10^\nu$.

Modelos de regressão baseados na distribuição COM-Poisson foram
propostos por \citet{Sellers2010}. Para $y_i$ observações
independentes do modelo COM-Poisson e
$\bm{x}_i^\top = (x_{i1}, x_{i2}, \ldots, x_{ip})$ um vetor de
covariáveis conhecidas, $i=1,2,\ldots,n$, o modelo de regressão
COM-Poisson é definido como
$$
Y_i \sim \text{CMP}(\lambda_i, \nu), \quad \text{em que} \quad
  \lambda_i = g^{-1}(\bm{x}_i^\top\bm{\beta}).
$$
Embora outras funções de ligação sejam possíveis, a função de ligação
logarítmica é, frequentemente, adotada.

Essa formulação tem o grande inconveniente de interpretabilidade, uma
vez que o modelo de regressão está associado a um parâmetro que não
representa a média. Ainda, para a distribuição COM-Poisson, em sua
parametrização original, $\lambda$ e $\nu$ são fortemente
intra-relacionados na função de verossimilhança
\citep{RibeiroJr2018}. Sob essas motivações \citet{Huang2017} e
\citet{RibeiroJr2018} propuseram reparametrizações para a média no
modelo COM-Poisson. Nesses casos, o modelo de regressão é especificado
como
$$
Y_i \sim \text{CMP}_\mu(\mu_i, \nu), \quad \text{em que} \quad
  \mu_i = g^{-1}(\bm{x}_i^\top\bm{\beta}),
$$
sendo $\text{CMP}_\mu(\mu_i, \nu)$ a distribuição COM-Poisson
reparametrizada para a média. Na abordagem proposta por
\Citeauthor{Huang2017}, o parâmetro $\mu_i$ da distribuição
$\text{CMP}_\mu(\mu_i, \nu)$ é obtido como raiz da equação
$$ \sum_{j=0}^{\infty} (j-\mu_i) \frac{\lambda_i^j}{(j!)^\nu} = 0, $$
enquanto que \Citeauthor{RibeiroJr2018}, obtém $\mu_i$ a partir
da aproximação para a média,
$$ \mu_i = h(\lambda_i, \nu) = \lambda_i^{1/\nu} - \frac{\nu -
  1}{2\nu}. $$
Em ambos os casos, $\mu_i$ e $\nu$ se mostram ortogonais e os parâmetros
$\bm{\beta}$ em um modelo de regressão têm a mesma interpretação do
modelo Poisson.

\section{Distribuição \textit{Gamma-Count}}
\label{sec:gamma-count}

A distribuição \textit{Gamma-Count} é uma generalização da distribuição
Poisson que resulta da relação da Poisson com a distribuição do tempo
entre eventos. Seja $\tau_k>0,\, k\in\mathbb{N}^*$ o tempo entre o
$(k-1)$ e o $k$-ésimo evento e $\vartheta_n$ o tempo de chegada do
$n$-ésimo evento. Então $\vartheta_n = \sum_{k-1}^{n}\tau_k$ para
$n=1,2,\ldots$. Seguindo \citet{Winkelmann1995}, seja $Y_T$ o número
total de eventos no intervalo $(0, T)$. Para $T$ fixo, $Y_T$ é uma
variável aleatória de contagem. Segue das definições de $Y_t$ e
$\vartheta_n$ que $Y_T < y$ e $\vartheta_y \geq T$ são eventos
equivalentes e portanto,
\begin{equation}
  \label{eqn:renewal}
  \begin{gathered}
    \Pr(Y_T < y) = \Pr(\vartheta_y \geq T) = 1 - \text{F}_y(T), \\
    \Pr(Y_T = y) = \Pr(Y_T < y) - \Pr(Y_T < y + 1) =
    \text{F}_y(T) - \text{F}_{y+1}(T),
  \end{gathered}
\end{equation}
em que $\text{F}_y(T)$ é a função distribuição acumulada de
$\vartheta_y$ e $T$ é a amplitude do intervalo de contagem,
frequentemente denotado como \textit{offset} nos modelos de regressão.

A \Cref{eqn:renewal} resulta uma distribuição de contagem a partir,
apenas, do conhecimento da distribuição do tempo entre eventos. Por
exemplo, a distribuição Poisson deriva da suposição de que $\tau_k$ são
exponencialmente distribuídos com parâmetro de taxa $\lambda$. Assim,
tem-se que $\vartheta_y$ segue uma distribuição de Erlang de parâmetros
$y$ e $\lambda$, e a função massa de probabilidade de $Y_T$ é obtida como
$$
\Pr(Y_T = y) = \text{F}_y(T) - \text{F}_{y+1}(T) =
  \sum_{j=0}^{y-1}\frac{\exp(-\lambda T)(\lambda T)^j}{j!} -
  \sum_{j=0}^{  y}\frac{\exp(-\lambda T)(\lambda T)^j}{j!} =
  \frac{\exp(-\lambda T)(\lambda T)^j}{y!},
$$
que, para $T=1$, é idêntica à \Cref{eqn:pmf-poisson}.

Para a distribuição \textit{Gamma-Count} GCT$(\alpha, \gamma)$ assume-se
distribuição gama, com de parâmetros $\alpha$ e $\gamma$, para os tempos
entre eventos $\tau_k$. Consequentemente, a distribuição de
$\vartheta_y$ também é gama, porém com parâmetros $y\alpha$ e
$\gamma$. Sua função massa de probabilidade de uma variável aleatória
$Y_t \sim \text{GCT}((\alpha, \gamma))$ é obtida por
\begin{equation}
  \label{eqn:pmf-gammacount}
  \Pr(Y_T = y) = \text{F}_y(T) - \text{F}_{n+1}(T) =
  \int_0^T \frac{\gamma^{y\alpha} t^{y\alpha - 1}}{\Gamma(y\alpha)
    \exp(\gamma t)} dt -
  \int_0^T \frac{\gamma^{(y+1)\alpha} t^{(y+1)\alpha - 1}}{
    \Gamma[(y+1)\alpha] \exp(\gamma t)} dt,
\end{equation}
que não tem forma fechada, a menos do caso particular $\alpha=1$ quando
a distribuição reduz-se à Poisson. Os momentos da distribuição também
não são obtidos em forma fechada. \citet{Winkelmann1995} mostra que
para intervalos de observação suficientemente grandes, $T\to\infty$,
sustenta-se que
$$ Y_T \overset{a}{\sim} \mathcal{N}\left (
  \frac{\gamma T}{\alpha}, \frac{\gamma T}{\alpha^2} \right ), $$
assim a razão média-variância assintótica é
$1/\alpha$. Consequentemente, a distribuição \textit{Gamma-Count} é
capaz de modelar superdispersão $(0 < \alpha < 1)$ e subdispersão
$(\alpha > 1)$.

Modelos de regressão \textit{Gamma-Count} foram propostos por
\citet{Winkelmann1995} e \citet{Zeviani2014}.  Para $y_i$
observações independentes do modelo $\text{GCT}(\alpha, \gamma_i)$ e
$\bm{x}_i^\top = (x_{i1}, x_{i2}, \ldots, x_{ip})$ um vetor de
covariáveis conhecidas, $i=1,2,\ldots,n$, o modelo de regressão
\textit{Gamma-Count} é definido como
$$
Y_i \sim \text{GCT}(\alpha, \gamma_i), \quad \text{em que} \quad
  \gamma_i = \alpha g^{-1}(\bm{x}_i^\top\bm{\beta}).
$$

A desvantagem desse modelo é que, embora assintoticamente
($T\to \infty$), $\gamma_i / \alpha$ represente a média das contagens,
em geral, modela-se a variável de contagem na escala do tempo entre
eventos, perdendo a interpretação dos parâmetros de regressão $\beta$.

\section{Distribuição Poisson generalizada}
\label{sec:poisson-generalizada}

A distribuição Poisson generalizada é resultante de uma forma limite da
distribuição binomial negativa generalizada e pode modelar sub e
superdispersão \citep{Zamani2012}. Seja $Y$ uma variável aleatória que
segue a distribuição Poisson generalizada GPo$(\lambda, \gamma)$, então
sua função massa de probabilidade é dada por
\begin{equation}
  \label{eqn:pmf-gpoisson0}
  \Pr(Y=y) =
  \begin{cases}
    \left [ \lambda (\lambda + y\gamma)^{y-1}
      \exp(-\lambda - y\gamma) \right ] / y!, & y =0, 1,2,\ldots \\
    0 & \text{para } y > m, \text{quando } \gamma < 0,
  \end{cases}
\end{equation}
em que $\lambda>0$, max$(-1, -\lambda/4) \leq \gamma \leq 1$ e $m$ é o
maior inteiro positivo tal que $\lambda + m\gamma >0$ quando $\gamma$ é
negativo \citep{Consul1992}. A média e a variância são dadas por
E$(Y) = \lambda(1-\gamma)^{-1}$ e Var$(Y) = \lambda(1-\gamma)^{-3}$,
respectivamente. A distribuição Poisson é um caso particular, quando
$\gamma=0$.

Para modelos de regressão, sugere-se a parametrização para a média
$$
\lambda = \frac{\mu}{1 + \alpha \mu} \quad \text{ e } \quad
\gamma = \frac{\alpha \mu}{ 1 + \alpha \mu},
$$
que substituindo em (\ref{eqn:pmf-gpoisson0}) leva à função massa de
probabilidade
\begin{equation}
  \label{eqn:pmf-gpoisson}
  \Pr(Y=y) = \left ( \frac{\mu}{1+\alpha \mu} \right )^y
    \frac{(1+\alpha y)^{y-1}}{y!}
    \exp \left [ - \mu \frac{(1 + \alpha y)}{( 1 + \alpha \mu)}
    \right ],
\end{equation}
denotada por $\text{GPo}_\mu(\mu, \alpha)$. Os momentos da distribuição
nessa parametrização são
\begin{equation}
  \label{eqn:moments-gpo}
\text{E}(Y) = \mu \quad \text{e} \quad
   \text{Var}(Y) = \mu (1 + \mu\alpha)^2,
\end{equation}
que garantem bastante flexibilidade à distribuição, uma vez que a
relação média--variância é determinada como uma função cúbica de $\mu$.

Para $y_i$ observações da distribuição Poisson generalizada e $\bm{x}_i$
um vetor conhecido de covariáveis, o modelo de regressão baseado na
distribuição Poisson generalizada é definido por
$$
Y_i \sim \text{GPo}_\mu(\mu_i, \alpha), \quad \text{em que} \quad
  \mu_i = g^{-1}(\bm{x}_i^\top\bm{\beta}),
$$
em que
$\alpha > \text{min}[ -\text{max}(y_i^{-1}), -\text{max}(\mu_i^{-1})]$,
quando $\alpha<0$.

Aplicações do modelo de regressão Poisson generalizada são pouco
reportadas na literatura. Embora bastante flexível, a grande dificuldade
desse modelo reside na complicada restrição do espaço paramétrico, que é
difícil de se incorporar, de forma eficiente, no processo de estimação.

\section{Distribuição Poisson-Tweedie}
\label{sec:poisson-tweedie}

Distribuições de variáveis na forma de contagens baseadas em
especificações hierárquicas da distribuição Poisson são comuns para
análise de dados superdispersos, e.g. distribuição binomial negativa. A
distribuição Poisson-Tweedie representa um caso geral dos modelos
Poisson com especificação hierárquica \citep[Seção 4.6]{Jorgensen1997}.

A distribuição Tweedie pertence à classe dos modelos exponenciais de
dispersão \citep{Jorgensen1997}. Seja E$(Y) = \mu$ e
Var$_p(Y) = \phi \mu^p$ então, $Y$ segue distribuição Tweedie
$\text{Tw}_p(\mu, \phi)$, em que $\mu \in \Omega_p$, $\phi > 0$ e
$p \in (-\infty, 0] \cup [1, \infty)$.  Note que o suporte da
distribuição depende de $p$, que age como um selecionador de
distribuições uma vez que as distribuições Gaussiana $(p = 0)$, Poisson
$(p = 1)$, non-central gamma $(p = 3/2)$, gamma $(p = 2)$ e Gaussiana
inversa $(p = 3)$ são casos particulares. A função densidade de
probabilidade da distribuição Tweedie não pode ser obtida em forma
fechada.

A distribuição Poisson-Tweedie PTw$_p(\mu, \phi)$ é resultante da
especificação hierárquica
\begin{equation}
  \label{eqn:espec-ptw}
  Y \mid Z \sim \text{Po}(Z) \quad \text{em que } \quad
    Z \sim \text{Tw}_p(\mu, \phi),
\end{equation}
que não tem forma fechada para função de probabilidade, exceto para
casos especiais. A esperança e a variância de uma variável aleatória
Poisson-Tweedie são obtidos por
\begin{equation}
  \label{eqn:moments-ptw}
  \begin{aligned}
  \text{E}(Y)   &= \text{E}[\text{E}(Y | Z)] = \mu \\
  \text{Var}(Y) &= \text{Var}[\text{E}(Y | Z)] +
    \text{E}[\text{Var}(Y | Z)] = \mu + \phi\mu^p.
  \end{aligned}
\end{equation}
Devido à flexibilidade da distribuição Tweedie, a Poisson-Tweedie também
tem importantes casos particulares que incluem as distribuições Hermite
$(p = 0)$, Neymann tipo-A $(p=1)$, Pólya-Aeppli $(p=1,5)$, binomial
negativa $(p=2)$ e Poisson--inversa gaussiana $(p = 3)$ \citep{Bonat2018}.

Pela definição (\ref{eqn:espec-ptw}), modelos Poisson-Tweedie só modelam
superdispersão. \citet{Bonat2018} estendem essa distribuição para
contemplar subdispersão, adotando apenas a especificação de momentos
baseada nas expressões em (\ref{eqn:moments-ptw}) e permitindo a
estimação de $\phi<0$. Essa abordagem é análoga aos modelos de
quase-verossimilhança \citep{Wedderburn1974}, e, a menos dos casos
particulares, não se conhece a distribuição completa de $Y$,
impossibilitando o cálculo de probabilidades, por exemplo.

Para observações $y_1, y_2, \ldots, y_n$ e seus respectivos vetores de
covariáveis $\bm{x}_i$, os modelos de regressão Poisson-Tweedie são
definidos por
$$
Y_i \sim \text{PTw}_p(\mu_i, \phi), \quad \text{em que} \quad
  \mu_i = g^{-1}(\bm{x}_i^\top\bm{\beta}).
$$
Em sua versão estendida, a especificação é dada por
$$
\text{E}(Y_i) = \mu_i = g^{-1}(\bm{x}_i^\top\bm{\beta})
  \quad \text{e} \quad
\text{Var}(Y_i) = \mu_i + \phi\mu_i^p.
$$
Note que para superdispersão ambas as especificações são equivalentes.

A grande motivação da utilização dessa distribuição é sua função de
variância, bastante flexível, em que a estimação de $p$ funciona como um
selecionador automático de distribuições.

\section{Estimação e inferência em modelos de regressão}
\label{sec:estimacao-e-inferencia}

Para as versões paramétricas dos modelos apresentados anteriormente, o
processo de estimação pode ser feito sob o paradigma de verossimilhança,
que é bastante intuitivo e fornece estimadores consistentes e
assintoticamente não viesados \citep{Pawitan2001}. Para $\bm{y}$ um
vetor $n\times 1$ conhecido de observações de uma mesma classe de
distribuições de probabilidade $f(y, \bm{\theta})$ e $\bm{X}$ uma matriz
$n \times p$ de delineamento ou matriz do modelo. A função de
verossimilhança e seu logaritmo são dados por
$$
\mathcal{L}(\bm{\theta}) = \mathcal{L}(\bm{\theta}; \bm{y}, \bm{X}) =
  \prod_{i=1}^n f(y_i; \bm{\theta}, \bm{x}_i)
\quad \text{e} \quad
\ell(\bm{\theta}) = \ell(\bm{\theta}; \bm{y}, \bm{X}) =
  \sum_{i=1}^n \log[f(y_i; \bm{\theta}_i, \bm{x}_i)],
$$
respectivamente, em que $\bm{\theta} \in \Theta$ é o conjunto de
parâmetros, desconhecido, que determina a distribuição e
$\bm{x}_i^\top = (x_{i1}, x_{i2}, \ldots, x_{ip})$ o vetor de
covariáveis da $i$-ésima observação. É interessante notar que para dados
discretos, em que
$f(y_i; \bm{\theta}; \bm{x}_i) \equiv \Pr(Y_i = y_i; \bm{\theta},
\bm{x}_i)$,
a função de verossimilhança representa a probabilidade de se observar
$\bm{y}$, dado que $\bm{\theta}$ é verdadeiro. O estimador de máxima
verossimilhança $\hat{\bm{\theta}}$ é obtido tal que
$\mathcal{L}(\hat{\bm{\theta}}; \bm{y}) = \underset{\bm{\theta} \in
  \Theta}{\text{max}}\,\mathcal{L}(\bm{\theta};
\bm{y})$.
Frequentemente, não é possível obter expressões em forma fechada para os
estimadores de máxima verossimilhança \citep{Bonat2017b}. Entretanto, é
usual assumir as estimativas de máxima verossimilhança como soluções
para as equações escore
\begin{equation}
  \label{eqn:score}
\mathcal{U}(\bm{\theta}) = \left (
  \frac{\partial \ell(\bm{\theta})}{\partial \theta_1},
  \frac{\partial \ell(\bm{\theta})}{\partial \theta_2},
  \ldots,
  \frac{\partial \ell(\bm{\theta})}{\partial \theta_p}
  \right )^\top = \bm{0},
\end{equation}
que são obtidas por métodos iterativos como Newton-escore, sendo a
matriz Hessiana
$$
\mathcal{H}(\theta) =
\begin{bmatrix}
    \frac{\partial \ell^2(\bm{\theta})}{\partial
      \theta_1^2} &
    \frac{\partial \ell^2(\bm{\theta})}{\partial
      \theta_1 \partial \theta_2} &
    \cdots &
    \frac{\partial \ell^2(\bm{\theta})}{\partial
      \theta_1 \partial \theta_p} \\
    \frac{\partial \ell^2(\bm{\theta})}{\partial
      \theta_2 \partial \theta_1} &
    \frac{\partial \ell^2(\bm{\theta})}{\partial
      \theta_2^2} &
    \cdots &
    \frac{\partial \ell^2(\bm{\theta})}{\partial
      \theta_2 \partial \theta_p} \\
    \vdots & \vdots & \ddots & \vdots \\
    \frac{\partial \ell^2(\bm{\theta})}{\partial
      \theta_p \partial \theta_1} &
    \frac{\partial \ell^2(\bm{\theta})}{\partial
      \theta_p \partial \theta_2} &
    \cdots &
    \frac{\partial \ell^2(\bm{\theta})}{\partial
      \theta_p^2}
  \end{bmatrix}
$$
obtida de forma analítica (e.g. Poisson). Quando
$\mathcal{H}(\bm{\theta})$ não tem forma fechada (e.g. COM-Poisson,
\textit{Gamma-Count}, Poisson generalizada), métodos que não necessitam
da especificação das derivadas de $\ell(\bm{\theta})$, como BFGS
\citep{Nocedal1995} são adotados. Nesses casos, a matriz Hessiana é
aproximada, geralmente, por diferenças finitas o que demanda um número
maior de avaliações do logaritmo da função de verossimilhança, tornando
o processo de estimação mais lento.

A inferência nos modelos paramétricos, ajustados por meio da maximização
da função de verossimilhança, baseia-se na distribuição assintótica dos
estimadores de máxima verossimilhança
$$
  \hat{\bm{\theta}} \overset{a}{\sim}
    \mathcal{N}(\bm{\theta}, \mathcal{I}(\bm{\theta}))
  \quad {e} \quad
  g(\hat{\bm{\theta}}) \overset{a}{\sim}
    \mathcal{N}(g(\bm{\theta}),
      \bm{G}^\top \mathcal{I}(\bm{\theta}) \bm{G}),
$$
em que $\mathcal{I}(\bm{\theta}) = -\mathcal{H}^{-1}$ é a matriz de
informação observada de Fisher, $g(\cdot)$ é uma função monótona e
diferenciável e
$\bm{G}^\top = (\partial g / \partial \beta_1, \ldots, \partial g
/ \partial \beta_p)^\top$ o vetor de derivadas primeiras de $g(\cdot)$.

Para a distribuição Poisson-Tweedie a estimação pelo método da máxima
verossimilhança é computacionalmente exigente, uma vez que as
probabilidades são dadas por uma integral intratável, exigindo o uso de
métodos numéricos. Considerando a especificação de momentos,
\citet{Bonat2018} propuseram o uso combinado das funções de estimação
quase-escore e de Pearson para estimação dos parâmetros de regressão e
de dispersão $(p, \phi)$, respectivamente.

Com essa pesquisa, a análise de dados na forma de contagens pretende ser
mais fiel ao processo gerador dos dados, levando a conclusões mais
assertivas nos experimentos conduzidos, principalmente, na área
agronômica. Com a apresentação e caracterização de distribuições
flexíveis para dados dessa natureza, definição dos modelos de regressão,
extensões para modelagem da dispersão e inclusão de efeitos aleatórios e
discussão dos métodos de estimação, a pesquisa almeja ser uma referência
metodologia para a estatística aplicada.

% ----------------------------------------------------------
% Chapter 1
% ----------------------------------------------------------
\chapter{Reparametrização dos modelos COM-Poisson}
\label{cha:reparcmp}

Nesse capítulo, propõe-se uma reparametrização do modelo COM-Poisson
baseada na expressão aproximada para média da distribuição, conforme
citado na \Cref{sec:com-poisson}.

\section{Reparametrização}

A reparametrização proposta, baseia-se na expressão aproximada para
média, \Cref{eqn:mean-aprox}. A reparametrização consiste na introdução
do parâmetro $\mu$ definido como
\begin{equation}
  \label{eqn:repar-cmp}
  \mu = h(\lambda, \nu) = \lambda^{1/\nu} - \frac{\nu - 1}{2\nu}
  \quad \Rightarrow \quad
  \lambda = h^{-1}(\mu, \nu) = \left (\mu +
    \frac{(\nu - 1)}{2\nu} \right )^\nu.
\end{equation}
O parâmetro de precisão da distribuição, denotado por $\nu$, é tomado na
escala do logaritmo para evitar a restrição no espaço paramétrico,
denota-se por $\phi$ esse novo parâmetro, ou seja,
$\phi = \log(\nu), \phi \in \mathbb{R}$. Esse parâmetro possui a mesma
interpretação de $\nu$, porém em uma escala diferente. Para $\phi < 0$
temos o caso de superdispersão, para $\phi > 0$ subdispersão e para
$\phi=0$ equidispersão, caso particular em que a COM-Poisson se resume à
Poisson.

Substituindo os novos parâmetros, definidos em (\ref{eqn:repar-cmp}), na
função massa de probabilidade (\ref{eqn:pmf-cmp}) tem-se
\begin{equation}
  \label{eqn:pmf-cmpmu}
  \Pr(Y=y \mid \mu, \phi) =
  \left ( \mu +\frac{ e^\phi-1}{2e^\phi} \right )^{ye^\phi}
  \frac{(y!)^{-e^\phi}}{Z(\mu, \phi)},
  \qquad y = 0, 1, 2, \ldots\, .
\end{equation}

Denota-se a distribuição COM-Poisson reparametrizada por
COM-Poisson$_\mu$. Na \Cref{fig:pmf-cmp} são apresentadas
distribuições de probabilidade para diferentes parâmetros ilustrando a
flexibilidade da distribuição. As distribuições para $\phi=0$
representam o caso particular Poisson.

\begin{figure}[!htb]

{\centering \includegraphics[width=\textwidth]{images/pmf-cmp-1} 

}

\caption[Probabilidades pela distribuição COM-Poisson$_\mu$ para diferentes parâmetros]{Probabilidades pela distribuição COM-Poisson$_\mu$ para diferentes parâmetros.}\label{fig:pmf-cmp}
\end{figure}



Uma avaliação da acurácia das aproximações apresentadas na
\Cref{eqn:mean-aprox}) e, consequentemente, da reparametrização é
apresentada na \Cref{fig:approx-plot} para valores de $\mu$ variando de
0 a 30 e diferentes níveis de dispersão ($-1,2 < \phi < 1$).  Os erros
quadráticos são obtidos como $[\mu - E(X)]^2$ e
$[\frac{\mu}{\nu} - V(X)]^2$, para a média e variância, respectivamente,
em que $E(X)$ e $V(X)$ são calculados numericamente usando a definição
de momentos.  As linhas tracejadas representam as restrições
$\nu \leq 1$ ou $\lambda > 10^\nu$.  Observa-se que a aproximação para a
média é acurada (valores menores que $0,03$), porém tem perda de
acurácia para combinações de $\nu$ pequenos ($<0,35$) com médias também
baixas ($\mu < 10$). Para a variância, \Cref{fig:approx-plot}(b), têm-se
que a aproximação $V[X] = \frac{\mu}{\nu}$, proposta por
\citet{Sellers2010}, não tem bom desempenho para valores pequenos
de $\nu$. A restrição que envolve a média da distribuição parece não ter
relevância. De forma geral, para média e variância, os valores de $\nu$
parecem ser mais influentes na acurácia do que os valores de $\mu$.

\begin{figure}[!htb]

{\centering \includegraphics[width=.8\textwidth]{images/approx-plot-1} 

}

\caption[Erros quadráticos das aproximações para (a) média e (b) variância]{Erros quadráticos das aproximações para (a) média e (b) variância.}\label{fig:approx-plot}
\end{figure}



Os resultados apresentados na Figura \Cref{fig:approx-plot} mostram que
a aproximação para o primeiro momento central da distribuição é acurada,
sendo assim, a reparametrização proposta adequada.

A fim de explorar a flexibilidade da distribuição COM-Poisson, além dos
formatos da distribuição apresentados na \Cref{fig:pmf-cmp}, obtêm-se os
índices de dispersão (DI), inflação de zeros (ZI) e cauda peada (HI),
que são respectivamente dados por
\begin{equation*}
\text{DI} = \frac{\text{Var}(Y)}{\text{E}(Y)}, \quad
\text{ZI} = 1 + \frac{\log \Pr(Y = 0)}{\text{E}(Y)}
  \quad \text{and} \quad
\text{HT} = \frac{\Pr(Y=y+1)}{\Pr(Y=y)}\quad \text{for} \quad y \to
\infty.
\end{equation*}
Os índices são definidos em relação a distribuição de Poisson. Então, O
índice de dispersão indica superdispersão quando $\text{DI} > 1$,
subdispersão quando $\text{DI} < 1$ e equidispersão quando
$\text{DI} = 1$. O índice de inflação de zero indica inflação para
$\text{ZI} > 0$, deflação para $\text{ZI} < 0$ e zeros de acordo com a
Poisson quando $\text{ZI} = 0$. Finalmente, o índice de cauda pesada
indica a que a distribuição é de cauda pesada para $\text{HT} \to 1$
quando $y \to \infty$.  O uso desses índices é discutido por
\citet{Bonat2017} para estudar a flexibilidade da distribuição
Poisson-Tweedie e por \citet{Puig2006} para descrever distribuição para
dados de contagem.

\begin{figure}[!htb]

{\centering \includegraphics[width=\textwidth]{images/indexes-plot-1} 

}

\caption[Índices para a distribuição COM-Poisson]{Índices para a distribuição COM-Poisson. (a) Relação média--variância, (b--d) Índices de dispersão, inflação de zeros e cauda pesada para diferentes parâmetros. As linhas tracejadas representam o caso especial Poisson.}\label{fig:indexes-plot}
\end{figure}



Os resultados apresentados na \Cref{fig:indexes-plot}, mostram que os
índices são levemente dependentes dos respectivos valores esperados e
tendem a estabilizar para $\mu$ grande. Consequentemente, a relação
média--variância \Cref{fig:indexes-plot}(a) é proporcional ao parâmetro
de dispersão $\phi$. Em termos de momentos, isso leva a uma
especificação indistinguível da abordagem via quasi-verossimilhança, em
que $\text{Var}(Y_i) = \sigma V(\mu_i) = \sigma
\mu$.
\Cref{fig:indexes-plot}(b) mostram que a distribuição é adequada para
modelar diferentes níveis de dispersão, como esperado. No conjunto de
parâmetros considerado, o maior valor de DI foi 4,21 e o menor
0,17. \Cref{fig:indexes-plot}(c) mostra que a distribuição pode lidar
com um limitado excesso de zeros, nos casos de superdispersão
($\phi < 0$). Por outro lado, para $\phi > 0$ (subdispersão) esta
distribuição é adequada para modelar deflação de zeros. Os índices de
cauda pesada indicam, em geral, que a distribuição é de caudas leves,
i.e $HT \to 0$ para $y \to \infty$.

\subsection{Estimação e Inferência}

Os modelos COM-Poisson padrão e reparametrizado são ajustados via
maximização da verossimilhança. Para uma amostra independente de
contagens $y_i$, $i=1,2\ldots,n$, as estimativas para
$\bm{\theta} = (\bm{\beta}\,, \phi)$ são obtidas pelos argumentos que
maximizam o logaritmo da função de verossimilhança
\begin{equation}
  \label{eqn:ll-rcmp}
  \ell(\bm{\beta}, \phi \mid \bm{y}) =
  e^\phi \left [
    \sum_{i=1}^n y_i
    \log \left( \mu_i + \frac{e^\phi-1}{2e^\phi} \right ) -
    \sum_{i=1}^n \log(y_i!) \right ] -
  \sum_{i=1}^n \log(Z(\mu_i, \phi)),
\end{equation}
em que $\mu_i = \exp(\bm{x}_i^t\bm{\beta})$, sendo $\bm{x}_i$ o vetor
$(x_{i1}, x_{i2}, \ldots x_{ip})$ de covariáveis da $i$-ésima
observação, e $(\bm{\beta}, \phi) \in \mathbb{R}^{p+1}$. A constante
$Z(\mu_i, \phi)$ é calculada como
$$
Z(\mu_i, \phi) = \sum_{j=0}^\infty \left [ \left (
    \mu_i + \frac{e^\phi - 1}{2e^\phi} \right )^{je^\phi}
  \frac{1}{(j!)^{e^\phi}} \right ]
$$
A avaliação do logaritmo da função de verossimilhança requer o cálculo
de uma série infinita para cada observação, o que torna sua computação
cara para regiões do espaço paramétrico cuja soma demora a convergir.

A estimação dos parâmetros requer a maximização numérica de
(\ref{eqn:ll-rcmp}). Como $\ell(\mu_i, \phi \mid \bm{y})$ não possui
derivada analítica, a maximação é realizada pelo algoritmo BFGS
\citep{Nocedal1995} que utiliza estimativas numéricas, via diferenças
finitas, para a matriz hessiana $\mathcal{H}(\bm{\theta})$. Os erros
padrão das estimativas são obtidos pela aproximação normal do logaritmo
da função de verossimilhança (método de Wald), fazendo $\sqrt{\bm{v}}$,
em que $\bm{v}$ são os elementos da diagonal da matriz
$-\hat{\mathcal{H}}^{-1}(\bm{\theta})$, estimada pelo algoritmo
BFGS. Intervalos de confiança para $\hat{\mu}_i$ são obtidos pelo método
delta \citep[p. 89]{Pawitan2001}.

\section{Estudo de simulação}

Nessa seção, são conduzidos estudos de simulação a fim de avaliar as
propriedades dos estimadores de máxima verossimilhança e a
ortogonalidade do modelo reparametrizado.

Nesse estudo, considera-se contagens variando de 3 a 27 de acordo com um
modelo de regressão com covariáveis contínua e categórica. A covariável
contínua ~($\bm{x}_1$) é gerada como uma sequência de $0$ a $1$ e
tamanho igual ao tamanho da amostra.  Analogamente, a covariável
categórica~($\bm{x}_2$) como uma sequência de três valores, em que cada
um é repetido $n/3$ vezes (arredondando quando necessário), em que $n$
denota o tamanho da amostra. Então, a esperança da variável aleatória
COM-Poisson é dada por
$\bm{\mu} = \exp(\beta_0 + \beta_1 \bm{x}_1 + \beta_{21} \bm{x}_{21} +
\beta_{22} \bm{x}_{22})$,
em que $\bm{x}_{21}$ e $\bm{x}_{22}$ são variáveis \textit{dummy}
representando os níveis de $\bm{x}_2$.  Os coeficientes de regressão são
fixados em $\beta_0 = 2$, $\beta_1 = 0.5$, $\beta_{21} = 0.8$ e
$\beta_{22} = -0.8$.

São estudos quatro diferentes cenários, considerando diferentes valores
para o parâmetro de dispersão $\phi = -1.6, -1.0, 0.0$ e $1.8$. Assim,
tem-se superdispersão forte e moderada, equidispersão e subdispersão,
respectivamente.  \Cref{fig:justpars} mostra a variação das médias das
contagens (à esquerda) e do índice de dispersão (à direita) para cada
para cenário considerado na simulação. Essa configuração, permite a
avaliação dos estimadores em situações extremas como contagens altas e
baixa dispersão, e contagens baixas e alta dispersão, mas também em
situação comuns como o caso de equidispersão.

\begin{figure}[!htb]

{\centering \includegraphics[width=\textwidth]{images/justpars-1} 

}

\caption[Contagens médias (left) and índices de dispersão (right) para cada cenário considerado no estudo de simulação]{Contagens médias (left) and índices de dispersão (right) para cada cenário considerado no estudo de simulação.}\label{fig:justpars}
\end{figure}



Para avaliar a consistência dos estimados, considera-se quatro
diferentes tamanhos de amostra: $50$, $100$, $300$ e $1000$; sendo
gerados $1000$ conjunto de dados em cada caso. Na \Cref{fig:bias-plot},
apresenta-se os vieses dos estimadores para cada cenário (combinação
entre os níveis de dispersão e tamanho da amostra) juntamente com
intervalos de confiança para os vieses, obtidos como média observada
mais ou menos $\Phi(0.975)$ vezes a média dos desvios padrão
observados. As escalas dos vieses são padronizadas pela média dos erros
padrão obtidos para a amostra de tamanho $50$.

\begin{figure}[!htb]

{\centering \includegraphics[width=\textwidth]{images/bias-plot-1} 

}

\caption[Distribuições dos viéses padronizadas (box-plots em cinza) e média com intervalos de confiança (segmentos em preto) de cada diferente tamanho de amostra e nível de dispersão]{Distribuições dos viéses padronizadas (box-plots em cinza) e média com intervalos de confiança (segmentos em preto) de cada diferente tamanho de amostra e nível de dispersão.}\label{fig:bias-plot}
\end{figure}



Os resultados na \Cref{fig:bias-plot}, mostram que para todos os níveis
de dispersão, ambos média dos vieses e desvios padrão tendem a $0$
conforme o tamanho da amostra aumenta. Os estimadores para os parâmetros
de regressão são não-viesados, consistentes e suas distribuições
empíricas são simétricas. Para o parâmetro de dispersão, o estimador é
assintoticamente não-viesado; em pequenas amostras o parâmetro é
superestimado e sua respectiva distribuição empírica é levemente
assimétrica à direita.

\begin{figure}[!htb]

{\centering \includegraphics[width=\textwidth]{images/coverage-plot-1} 

}

\caption[Taxas de cobertura dos intervalos de confiança obtidos por aproximação quadrática da verossimilhança para diferentes tamanhos de amostra e níveis de dispersão]{Taxas de cobertura dos intervalos de confiança obtidos por aproximação quadrática da verossimilhança para diferentes tamanhos de amostra e níveis de dispersão.}\label{fig:coverage-plot}
\end{figure}



Na \Cref{fig:coverage-plot} são apresentadas as taxas de cobertura
empíricas dos intervalos de confiança assintóticos, baseados na matriz
de informação de Fisher observada. Os resultados mostram que as taxas
de cobertura são próximas ao nível nominal de 95\% para amostras maiores
que $100$ em todos os cenários. Para o parâmetro de dispersão as taxas
de cobertura são levemente menores que o valor nominal; entretanto,
tornam-se próximos do nível nominal conforma aumenta o tamanho de
amostra. O pior cenário é quando se tem uma amostra pequena de contagens
fortemente superdispersas.

Para avaliar a propriedade de ortogonalidade, foram obtidas as matrizes
de covariância entre os estimadores de máxima verossimilhança,
$\hat{\bm{\theta}} = (\hat{\bm{\beta}}, \phi)$, obtidas da matriz de
informação observada,
Cov$(\hat{\bm{\theta}}) = \mathcal{I}^{-1}(\bm{\theta})$.
Na \Cref{fig:ortho-plot}, apresenta-se as covariâncias entre os
parâmetros de regressão e dispersão para cada cenário, na escala da
correlação. As correlações são praticamente zero em todos os casos,
evidenciando a ortogonalidade do modelo COM-Poisson
reparametrizado.

\begin{figure}[!htb]

{\centering \includegraphics[width=\textwidth]{images/ortho-plot-1} 

}

\caption[Correlações empíricas entre os estimadores dos parâmetros de regressão e dispersão para cada tamanho de amostra e nível de dispersão]{Correlações empíricas entre os estimadores dos parâmetros de regressão e dispersão para cada tamanho de amostra e nível de dispersão.}\label{fig:ortho-plot}
\end{figure}



\begin{figure}[!htb]

{\centering \includegraphics[width=\textwidth]{images/ortho-surf-1} 

}

\caption[Gráficos de contorno das superfícies de deviance sob a parametrização original e proposta para quatro conjunto de dados simulados de tamanho 1000]{Gráficos de contorno das superfícies de deviance sob a parametrização original e proposta para quatro conjunto de dados simulados de tamanho 1000. As elipses representam as regiões de confiança de 90, 95 and 99\%, as linhas tracejadas são as estimativas de máxima verossimilhança e os pontos são os valores dos parâmetros usados na simulação.}\label{fig:ortho-surf}
\end{figure}



Para ilustrar a ortogonalidade no modelo COM-Poisson$_\mu$, são
apresentados na \Cref{fig:ortho-surf} os gráficos de contornos da
superfície de deviance dos modelos COM-Poisson reparametrizado e sob a
parametrização original para quatro conjuntos de dados simulados de
tamanho 1000 com $\mu=5$ e diferentes valores para o parâmetro de
dispersão. Os formatos da superfície de deviance mostram que a
parametrização proposta é melhor no aspecto computacional para ajuste do
modelo e inferência assintótica (baseada na distribuição normal).

% ----------------------------------------------------------
% ELEMENTOS PÓS-TEXTUAIS
% ----------------------------------------------------------
\postextual

% ----------------------------------------------------------
% Referências bibliográficas
% ----------------------------------------------------------
\bibliography{../bibliography}

\end{document}
